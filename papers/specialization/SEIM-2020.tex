\documentclass[conference]{IEEEtran}
\IEEEoverridecommandlockouts
% The preceding line is only needed to identify funding in the first footnote. If that is unneeded, please comment it out.
\usepackage{cite}
\usepackage{amsmath,amssymb,amsfonts}
\usepackage{algorithmic}
\usepackage{graphicx}
\usepackage{textcomp}
\usepackage{xcolor}
\usepackage[utf8]{inputenc}
\usepackage[russian]{babel}
\def\BibTeX{{\rm B\kern-.05em{\sc i\kern-.025em b}\kern-.08em
    T\kern-.1667em\lower.7ex\hbox{E}\kern-.125emX}}
\begin{document}


\newcommand{\miniKanren}{\textsc{miniKanren}}


\title{Особенности Специализации Реляционных Программ}

\author{\IEEEauthorblockN{1\textsuperscript{st} Мария Куклина}
\IEEEauthorblockA{\textit{dept. name of organization (of Aff.)} \\
\textit{name of organization (of Aff.)}\\
City, Country \\
email address or ORCID}
\and
\IEEEauthorblockN{2\textsuperscript{nd} Екатерина Вербицкая}
\IEEEauthorblockA{\textit{dept. name of organization (of Aff.)} \\
\textit{name of organization (of Aff.)}\\
City, Country \\
email address or ORCID}

}

\maketitle

\begin{abstract}
    
\end{abstract}

\begin{IEEEkeywords}
Специализаия, реляционное программирование, суперкомпиляция, частичная дедукция 
\end{IEEEkeywords}

\section{Введение}

Вводная про реляционное программирование. (чем полезно в народном хозяйстве)

Вводная про \miniKanren. Как выглядят программы, про запуск в разных направлениях.

Про решение задач поиска силами распознавателя на \miniKanren.~\cite{lozov2019relational}

Про необходимость улучшать специализаторы. 

В этой статье мы делаем то-сё, пятое-десятое, измеряем, сравниваем и описываем, что делать дальше. 



\section{Специализация}

Специализация (или частичные вычисления) это вот это.~\cite{jones1993partial}

Есть много разных вариантов специализаторов, в том числе суперкомпиляция~\cite{soerensen1996positive} для функциональных языков и конъюнктивная частичная дедукция~\cite{de1999conjunctive} для логических. 

\subsection{Суперкомпиляция}

\subsection{Конъюнктивная частичная дедукция}

\section{Подходы для специализации \miniKanren}

Мы адаптировали конъюнктивную частичную дедукцию и суперкомпиляцию для \miniKanren, при этом для каждой из них сделали много разных модификаций. 

В этом разделе мы их опишем и сравним. 


\subsection{Подход 1}

\subsection{Подход 2}

\subsection{...}

\subsection{Подход n}

\subsection{Сравнение подходов и выводы}


\section{Заключение}

Итого, мы реализовали разные подходы, сравнили их и получили следующие результаты. 


\section*{Acknowledgment}

Выражаем благодарность Дмитрию Юрьевичу Булычеву и Даниилу Андреевичу Березуну за плодотворные дискуссии и конструктивную критику.
 

\bibliographystyle{IEEEtran.bst}
\bibliography{IEEEbib.bib}
\end{document}
