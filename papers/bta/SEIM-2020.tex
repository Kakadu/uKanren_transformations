\documentclass[conference]{IEEEtran}
\IEEEoverridecommandlockouts
% The preceding line is only needed to identify funding in the first footnote. If that is unneeded, please comment it out.
\usepackage{cite}
\usepackage{amsmath,amssymb,amsfonts}
\usepackage{algorithmic}
\usepackage{graphicx}
\usepackage{textcomp}
\usepackage{xcolor}
\usepackage[utf8]{inputenc}
\usepackage[russian]{babel}
\def\BibTeX{{\rm B\kern-.05em{\sc i\kern-.025em b}\kern-.08em
    T\kern-.1667em\lower.7ex\hbox{E}\kern-.125emX}}
\begin{document}


\newcommand{\miniKanren}{\textsc{miniKanren}}
\newcommand{\mercury}{\textsc{mercury}}


\title{Анализ времени связывания для реляционных программ}

\author{\IEEEauthorblockN{1\textsuperscript{st} Ирина Артемьева}
\IEEEauthorblockA{\textit{dept. name of organization (of Aff.)} \\
\textit{name of organization (of Aff.)}\\
City, Country \\
email address or ORCID}
\and
\IEEEauthorblockN{2\textsuperscript{nd} Екатерина Вербицкая}
\IEEEauthorblockA{\textit{dept. name of organization (of Aff.)} \\
\textit{name of organization (of Aff.)}\\
City, Country \\
email address or ORCID}

}

\maketitle

\begin{abstract}
    
\end{abstract}

\begin{IEEEkeywords}
Реляционное программирование, анализ времени связывания
\end{IEEEkeywords}

\section{Введение}

Вводная про реляционное программирование. (чем полезно в народном хозяйстве)

Вводная про \miniKanren. Как выглядят программы, про запуск в разных направлениях.

Про решение задач поиска силами распознавателя на \miniKanren.~\cite{lozov2019relational}

Про то, что даже имея хороший специализатор~\cite{jones1993partial}, сложно достичь супер-быстрых программ, поэтому мы решили транслировать в функциональные. 

\miniKanren --- язык хитрый, и трансляция в функциональный язык должна осуществляться с учетом направления. При этом направление влияет в том числе на порядок вычислений. 

Для определения порядка связывания имен в программах давно придумали анализ времени связывания. В \mercury он давно есть~\cite{vanhoof2004binding}, а для \miniKanren его что-то не сделали, но мы исправили это недоразумение.

В этой статье мы делаем то-сё, пятое-десятое, измеряем, сравниваем и описываем, что делать дальше. 



\section{\miniKanren}

\miniKanren это такой язык. 

Вот его конструкции, вот так он исполняется (в разные стороны по-разному!). 

Что нужно для трансляции его в функциональный язык (с выводом: было бы неплохо иметь bta).


\section{Анализ времени связывания для \miniKanren}

Вот так мы его делаем. 

Вот поэтому считаем, что это работает. Вот тут семантика для \miniKanren~\cite{rozplokhas2019certified}.

\section{Заключение}

Итого, мы разработали анализ времени связывания для \miniKanren. 

Доказали, что он работает. 

В будущем еще и в транслятор вставим, вот житуха начнется-то. 


\section*{Acknowledgment}

Выражаем благодарность Дмитрию Юрьевичу Булычеву и Даниилу Андреевичу Березуну за плодотворные дискуссии и конструктивную критику.
 

\bibliographystyle{IEEEtran.bst}
\bibliography{IEEEbib.bib}
\end{document}
